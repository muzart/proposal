
Collaborative modeling as a field of both practice and research has developed over the last decades. Within the field of system dynamics, modelers started to involve client groups in the modeling process since the late 1970s [11]. Since that time, various other modeling schools have adopted the notion of collaborative modeling and found approaches to involve stakeholders in their own modeling efforts [10].

	Bang et al. on their article [20] represented CoDesign – collaborative software modeling framework. CoDesign's main contribution is an extensible conflict detection framework for collaborative modeling. CoDesign utilizes an event-based architecture in which highly-decoupled Components-different instances of CoDesign - exchange messages via implicit invocation, allowing flexible system composition and adaptation. In the article modeling-level conflicts categorized into three classes based on the rules that the system modeling events violate: synchronization, syntactic and semantic conflicts. 

Kuryazov et al. on their article [5] represented Delta Operations Language (DOL), which is utilized to model difference representations. DOL is a set of domain specific languages to model difference representation in terms of delta language operations. In order to derive a specific DOL for a specific modeling language, the meta-model of a modeling language is required. A DOL Generator used to generate a specific DOL for a certain modeling language using the meta-model. Then, a specific DOL is fully capable of representing all differences between subsequent versions of the instance models in terms of operation-based DOL in modeling deltas. The operations in modeling deltas are referred to as Delta Operations. DOL aims at supporting several DOL-services, which can access and reuse delta operations. These operative services make the DOL-based modeling deltas quite handy in various application areas and enable application areas to utilize the DOL-based modeling deltas. A specific DOL is derived for a specific modeling language and several DOL-services. DOL-services provide means to manage DOL-based deltas. Services relying on representation of the DOL approach are applied to model versioning and model history analysis, using state-based difference calculation [5]. DOL can.be utilized as a strong basis for organizing the collaborative modelling process. The main goals of representing DOL are: implementation of understandable high-level modeling language and organizing collaborative modeling process on basis of it. But, as authors mentioned, DOL-services list should be extended by adding a runtime operation recording service, difference merger service and synchronizer service for organize real-time collaborative modeling process based on DOL.
